\section{Metod}

Metodkapitlet beskriver hur man lägger upp arbetet. Vilket bland annat omfattar arbetsgång, design av experiment och användning av olika metoder för datainsamling. I idealfallet ska ett metodkapitel vara så utförligt att vem som helst med vissa baskunskaper inom området ska kunna utföra arbetet på det sätt som beskrivs och nå samma resultat. Att beskriva metoden är viktigt för att uppdragsgivaren ska kunna bedöma om målet går att nå på det föreskrivna sättet. Därför är det också viktigt att förklara i det här kapitlet varför den valda metoden ger ett tillförlitligt resultat.
\\[5mm]

För att kunna klara av detta arbetet behövs en faktainsamling att göras. Även kvalitativ diskussion mellan grupp medlemmarna samt handledare på företget och Chalmers. I faktainsalmigen kommer internet sökning att användas, då det finns bra och gratis information om programmering i olika språk. Informativa videos kommer också ge oss ökad förståelse innan arbetet kan dra igång. Då det finns många open soruces online där man kan ta del av andras kunskap är detta ett bra sätt att lära sig på. 

\\[5mm]
När väl faktainsamlingen är gjord kan man börja med att programmera ett program som kommer jämföra filer.För att automatiserad kontroll kommer kunna vara genomförbar. Under tiden kommer en fortsatt diskussion ske mellan de iblandade, för att kunna få inputs under arbetsgången. Genom att man succesivt testar programmeringen och sparar denna i versions filer under programmeringens gång kommer man kunna få ett bra slutresultat. 

